\documentclass[12pt]{beamer}

\usepackage[T2A]{fontenc}
\usepackage[utf8]{inputenc}
\usepackage[russian]{babel}
\usepackage{amsthm, amsmath, amssymb}
\usepackage{hyperref}
\usepackage{datetime}
\usepackage{cmap}
\usepackage{enumerate}
\usepackage{color}
\usepackage{picture}
\usepackage{graphicx}
\usepackage{tikz}
\usepackage{xcolor}
\usetikzlibrary{positioning,shadows,arrows}

\usepackage{bold-extra}

\def\EPS{\varepsilon}
\def\SO{\Rightarrow}
\def\EQ{\Leftrightarrow}
\def\t{\texttt}

\usetheme{Warsaw}

\let\Tiny=\tiny
\useoutertheme{infolines}

\tikzset {
    fact/.style={rectangle, draw=none, rounded corners=1mm, fill=blue, drop shadow,
        text centered, anchor=north, text=white},
    new/.style={circle, draw=none, fill=orange, circular drop shadow,
        text centered, anchor=north, text=white},
    state/.style={circle, draw=none, fill=red, circular drop shadow,
        text centered, anchor=north, text=white},
    leaf/.style={rectangle, draw=black,
    minimum width=0.5em, minimum height=0.5em},
    cur/.style={circle, draw=none, fill=green, circular drop shadow,
        text centered, anchor=north, text=black},
    level distance=1.0cm, anchor=south
}

\begin{document}

\title{Расписание занятий для АУ}

\author[]{
    Кравченко Дмитрий \\
    Розплохас Дмитрий \\ 
}
\institute[]{Санкт-Петербургский Академический университет}
\date{18 декабря 2015 года}

\frame{\titlepage}

\begin{frame}{Цели и задачи}

    \begin{itemize}

        \item <1-> Приложение для просмотра расписания занятий

        \item <2-> Просмотр объявлений
        
        \item <3-> Сервер для загрузки расписания

        \end{itemize}

\end{frame}

\begin{frame}{Технические подробности}

    \begin{itemize}
    
        \item <1-> Проект написан на Java

        \item <2-> Для удобства работы в команде использовался git

        \item <3-> Работа с сетью

        \item <4-> Внешнии библеотеки: Google App Engine и ко

    \end{itemize}

\end{frame}

\begin{frame}{Результаты}
    
    \begin{itemize}
    
        \item <1-> Вкладка Расписание

        \item <2-> Вкладка Результаты

        \item <3-> Сервер, поддерживающий загрузку расписания и ссылок на результаты

    \end{itemize}

\end{frame}

\begin{frame}{Приложение}
    
    \begin{itemize}

    \item <1-> Если файл уже загружен,
               то Wi-fi совсем не нужен!

    \item <2-> Файлы сохраняются в кэше

    \end{itemize}

\end{frame}

\begin{frame}{Сервер}
    
    \begin{itemize}

    \item <1-> Загружать расписание можно только в формате xml

    \item <2-> Возможность добавлять и удалять ссылки на результаты

    \end{itemize}

\end{frame}

\begin{frame}{Недоделки}
    
    \begin{itemize}

    \item <1-> Нет просмотра уведомлений

    \item <2-> Нет контактов

    \item <2-> Нет нотификаций

    \end{itemize}

\end{frame}

\begin{frame}{Ссылки}
    
    \begin{itemize}

        \item <1-> Видео приложения: https://www.youtube.com/watch?v=Nd5dpVuJPBM\&feature=youtu.be

        \item <2-> Репозиторий: https://github.com/auTimetable/auTimetable

    \end{itemize}

\end{frame}

\begin{frame}{Конец}

    Вопросы?

\end{frame}

\end{document}